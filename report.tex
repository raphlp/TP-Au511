\documentclass{article}

\usepackage{graphicx}
\usepackage{geometry}
\usepackage{amsmath,amssymb}
\usepackage{float}
\usepackage[T1]{fontenc}
\usepackage{hyperref}

\geometry{left=2.5cm,right=2.5cm,top=2.5cm,bottom=2.5cm}

\begin{document}

% ================== TITLE PAGE ==================
\begin{titlepage}
\begin{flushleft}
{\large LAUPIES Rapha\"el}\\[0.2cm]
{\large CHARDON DU RANQUET Quentin\"el}\\[0.2cm]
{\large IPSA -- A5}\\[0.2cm]
{\large AU511 -- Control of Aircraft}
\end{flushleft}

\vspace*{\fill}
\centering
{\huge Home Work 2025--2026}\\[0.6cm]
{\Large Extended Kalman Filter \& System Identification}\\[0.8cm]

\includegraphics[width=0.65\textwidth]{logoipsa.png}

\vspace{1cm}
{\large Supervisor: Yamine SELLAMI}\\[0.4cm]
{\small June 2025}

\vspace*{\fill}
\end{titlepage}

\tableofcontents
\newpage


% ================== INTRODUCTION ==================
\section{Introduction}
This practical work focuses on the modelling, analysis and control of the
longitudinal dynamics of a fighter-class aircraft. The objective is to design
and evaluate a complete longitudinal autopilot architecture, starting from the
definition of an in-flight operating point up to a multi-phase flight management
simulation. Classical control techniques are employed, including state-space
modelling, modal analysis, feedback loop design and saturation handling.

% ================== PART 1 ==================
\section{In-Flight Operating Point}

\subsection{Flight Conditions}
The operating point is defined by a prescribed Mach number and altitude. The
corresponding atmospheric properties are obtained using the US Standard
Atmosphere model.

\subsection{Equilibrium Computation}
The longitudinal equilibrium is determined by solving the nonlinear balance of
forces and moments. The equilibrium angle of attack, elevator deflection and
thrust force are obtained through an iterative numerical procedure.

% ================== PART 2 ==================
\section{Aircraft Longitudinal Model}

\subsection{State-Space Representation}
A linearised small-signal model is derived around the equilibrium point. The
state vector includes velocity, flight-path angle, angle of attack, pitch rate,
pitch angle and altitude. The elevator deflection is used as control input.

\subsection{Model Parameters}
The aerodynamic derivatives and inertial parameters used in the model are
presented and discussed.

% ================== PART 3 ==================
\section{Open-Loop Dynamic Analysis}

\subsection{Modal Analysis}
The longitudinal modes of the aircraft (short-period and phugoid) are identified
through eigenvalue analysis. Natural frequencies and damping ratios are
computed.

\subsection{Time-Domain Responses}
Step responses of the open-loop system are analysed to characterise the dynamic
behaviour of the aircraft without control.

% ================== PART 4 ==================
\section{Pitch Rate Control Loop}

\subsection{Design of the $q$-Feedback Loop}
A pitch-rate feedback loop is designed in order to stabilise the short-period
mode and improve damping. The control gain is selected according to damping
ratio specifications.

\subsection{Closed-Loop Analysis}
The closed-loop poles and step responses are analysed and compared to the
open-loop case.

% ================== PART 5 ==================
\section{Washout Filter}

\subsection{Motivation}
A washout filter is introduced in the pitch-rate feedback loop to preserve the
steady-state behaviour of the angle of attack while maintaining dynamic
improvements.

\subsection{Performance Comparison}
The responses obtained with and without washout filtering are compared in both
time and steady-state domains.

% ================== PART 6 ==================
\section{Flight-Path Angle Control}

\subsection{Gamma Control Loop}
Assuming perfect speed regulation, a flight-path angle feedback loop is designed
on top of the pitch-rate loop.

\subsection{Dynamic Performance}
The resulting closed-loop dynamics are analysed and validated through time
responses.

% ================== PART 7 ==================
\section{Altitude Control}

\subsection{Altitude Loop Design}
An altitude control loop is introduced, including the dynamics of the altitude
sensor modelled as a first-order system.

\subsection{Closed-Loop Behaviour}
The altitude response is evaluated in terms of overshoot, settling time and
tracking accuracy.

% ================== PART 8 ==================
\section{Command Saturation}

\subsection{Load Factor Constraint}
A limitation on the commanded flight-path angle is imposed based on a maximum
allowable load factor.

\subsection{Computation of Saturation Limits}
The maximum admissible command is computed using a nonlinear relationship
between flight-path angle and angle of attack.

% ================== PART 9 ==================
\section{Flight Management Simulation}

\subsection{Multi-Phase Scenario}
A complete flight sequence is simulated, including climb, cruise, descent and
final flare phases.

\subsection{Simulation Results}
The time evolution of altitude, flight-path angle and pitch angle is analysed and
commented.

% ================== CONCLUSION ==================
\section{Conclusion}
This practical work demonstrates the complete design of a longitudinal autopilot
for a high-speed aircraft. The proposed control architecture satisfies stability
and performance requirements while remaining consistent with physical
constraints. The results validate the effectiveness of classical control
approaches for aircraft longitudinal dynamics.

\end{document}